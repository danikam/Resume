
\section{Experience}

\outerlist{

\entrybig
	{\textbf{University of Victoria}}{Victoria, Canada}
	{Graduate Research Assistant (Doctoral)}{Sept. 2018 - July 2022}
\innerlist{
	\entry{Authorship qualification work and PhD dissertation with the UVic ATLAS group}
	\entryextra{Developed containerized cloud computing infrastructure as part of authorship qualification work for the ATLAS collaboration. Served as primary analyst and contact person for a 7-person international team of scientists to perform a sophisticated analysis of particle collision data from the Large Hadron Collider, which searched for evidence of dark matter production in the high-energy collisions.}
}

\entrybig
	{\textbf{ATLAS Collaboration}}{Remote}
	{Analysis Preservation Contact}{Feb. 2020 - Oct. 2021}
\innerlist{
	\entry{Analysis preservation contact person for the ATLAS collaboration (5,000 members)}
	\entryextra{Developed numerous analysis preservation workflows using a framework developed for high-energy physics that incorporates Docker, GitLab CI and Kubernetes. Provided technical assistance, liaison, and central documentation to support analysis teams with the development of their own analysis preservation frameworks. Organized hands-on training events to familiarize analysts with the tools involved with analysis preservation.}
}

\entrybig
	{\textbf{University of British Columbia}}{Vancouver, Canada}
	{Graduate Research Assistant (Master's)}{Sept. 2016 - Aug. 2018}
\innerlist{
	\entry{DAQ development and detector calibration}
	\entryextra{Designed a real-time `baseline control? algorithm to maintain signal integrity for data collected by cryogenically-cooled solid-state SuperCDMS detectors currently being installed at the SNOLAB facility in Sudbury, Ontario. Analyzed calibration data in collaboration with a 8-person international team of scientists to improve the modelling of ionization yield from nuclear recoil events in solid-state Ge and Si detectors.}
}

}
